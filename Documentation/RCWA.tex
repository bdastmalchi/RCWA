
%----------------------------------------------------------------------------------------
%	PACKAGES AND DOCuMENT CONFIGuRATIONS
%----------------------------------------------------------------------------------------

\documentclass{article}


\usepackage[letterpaper, portrait, margin=1.0in]{geometry}

\usepackage{siunitx} % Provides the \SI{}{} and \si{} command for typesetting SI units
\usepackage{graphicx} % Required for the inclusion of images
\usepackage{natbib} % Required to change bibliography style to APA
\usepackage{amsmath} % Required for some math elements 
\usepackage{isomath} % Required for some math elements 
\usepackage{mathrsfs,mathabx}
\usepackage{empheq}
\usepackage[most]{tcolorbox}
\usepackage{mathtools,array,booktabs}

\newtcbox{\mymath}[1][]{%
    nobeforeafter, math upper, tcbox raise base,
    enhanced, colframe=blue!30!black,
    colback=blue!30, boxrule=0.1pt,
    #1}


\setlength\parindent{0pt} % Removes all indentation from paragraphs

\renewcommand{\labelenumi}{\alph{enumi}.} % Make numbering in the enumerate environment by letter rather than number (e.g. section 6)

%\usepackage{times} % uncomment to use the Times New Roman font

%----------------------------------------------------------------------------------------
%	DOCuMENT INFORMATION
%----------------------------------------------------------------------------------------

\title{Scattering from a Sheet of Periodic Conductivity} % Title

\author{Babak \textsc{Dastmalchi}} % Author name

%\date{\today} % Date for the report



\begin{document}

\maketitle % Insert the title, author and date




% If you wish to include an abstract, uncomment the lines below
% \begin{abstract}
% Abstract text
% \end{abstract}

%==========================================
%==========================================
\section{Notation}

\begin{itemize}
\item Upper and lower media are denoted by substrate $I$ and $II$, respectively. 
\item $R_i$ and $T_i$ respectively denote normalized reflected and Transmitted field amplitudes for the $i$th order.
\item Physics notation has been chosen, i.e. $e^{-\jmath\left(\omega t - \vec{k}\dot\vec{r}\right) }$
\item $\mu=\mu_0$ for all media
\item For the TE polarization $\mathbf{E}=E_y\mathbf{\hat{y}}$ and $\mathbf{H}=H_x\mathbf{\hat{x}}+H_z\mathbf{\hat{z}}$
\item parallel component of k-vector for i-th diffracted order is: $\beta_i=k_0 n_I \sin(\theta) - \frac{2\pi}{\Lambda} i$, where $k_0$ is the free space k-vector, $n_I$ is the refractive index for incident medium, and $\Lambda$ is the periodicity.
\end{itemize}





%==========================================
%==========================================
\section{Scattering Problem}

%==========================================
%==========================================
\subsection{Tangential Fields at Media $I$ and $II$}

\begin{empheq}[box={\mymath[colback=white!30,drop lifted shadow, sharp corners]}]{equation}
\label{eq:tang_e} 
\begin{array}{l}
E_{Iy} = e^{\jmath\left(\beta_0 x+k_{I,z0}z\right)}+\sum\limits_i R_i e^{\jmath\left(\beta_i x-k_{I,zi}z\right)}\\
E_{IIy} = \sum\limits_i T_i e^{\jmath\left[\beta_i x+k_{II,zi}(z-d)\right]}\\
\end{array}	
\end{empheq}

\begin{equation*} 
\label{eq:tang_h} 
\begin{array}{l}
H_{Ix} = \frac{\jmath}{\mu_0\omega}\partial_zE_{Iy}=\frac{\jmath}{\mu_0\omega}\left[ \jmath k_{I,z0}e^{\jmath\left(\beta_0 x+k_{I,z0}z\right)}+\sum\limits_i R_i \left(-\jmath k_{I,zi}\right)e^{\jmath\left(\beta_i x-k_{I,zi}z\right)} \right]\\
H_{IIx} = \frac{\jmath}{\mu_0\omega}\partial_zE_{IIy}=\frac{\jmath}{\mu_0\omega}\{ \sum\limits_i T_i \left(\jmath k_{II,zi}\right) e^{\jmath\left[\beta_i x+k_{II,zi}(z-d)\right]} \}\\
\end{array}	
\end{equation*}

assuming $Y_{I,i}=\frac{-1}{\mu_0\omega}k_{I,zi}$ and $Y_{II,i}=\frac{-1}{\mu_0\omega}k_{II,zi}$

\begin{empheq}[box={\mymath[colback=white!30,drop lifted shadow, sharp corners]}]{equation}
\label{eq:tang_h} 
\begin{array}{l}
H_{Ix} = Y_{I,0}e^{\jmath\left(\beta_0 x+k_{I,z0}z\right)} - \sum\limits_i R_i  Y_{I,i} e^{\jmath\left(\beta_i x-k_{I,zi}z\right)} \\
H_{IIx} = \sum\limits_i T_i  Y_{II,i} e^{\jmath\left[\beta_i x+k_{II,zi}(z-d)\right]} \\
\end{array}	
\end{empheq}

%==========================================
%==========================================
\subsection{Field expansion inside the grating region}
Fields inside the gratings can be expanded in Fourier space harmonic bases

\begin{empheq}[box={\mymath[colback=white!30,drop lifted shadow, sharp corners]}]{equation}
\label{eq:tang_h} 
\begin{array}{l}
E_{y} = \sum\limits_i S_i(z) e^{\jmath \beta_i x} \\
H_{x} = \sum\limits_i U_i(z) e^{\jmath \beta_i x} \\
\end{array}	
\end{empheq}

and

\begin{empheq}[box={\mymath[colback=white!30,drop lifted shadow, sharp corners]}]{equation}
\label{eq:tang_h} 
\begin{array}{l}
S_i(z) = \sum\limits_m  \left( g_{i,m}^{+} e^{\jmath \gamma_m z} + g_{i,m}^{-} e^{-\jmath \gamma_m (z-d)} \right) \\
U_i(z) = \frac{-1}{\mu_0\omega} \sum\limits_m  \left(  \gamma_m g_{i,m}^{+} e^{\jmath \gamma_m z} - \gamma_m g_{i,m}^{-} e^{-\jmath \gamma_m (z-d)} \right)
\end{array}	
\end{empheq}

where $\gamma_m$s are the z components of the k-vector in the grating medium

%==========================================
%==========================================
\subsection{Boundary condition at z=0}

\begin{equation*} 
\label{eq:rcwa_Maxwell_1} 
\begin{array}{l}
e^{\jmath\beta_0 x}+\sum\limits_i R_i e^{\jmath\beta_i x} = \sum\limits_i e^{\jmath \beta_i x} \sum\limits_m  \left( g_{i,m}^{+} + g_{i,m}^{-} e^{\jmath \gamma_m d} \right)  \\\\
Y_{I,0}e^{\jmath\beta_0 x} - \sum\limits_i R_i  Y_{I,i} e^{\jmath\beta_i x} = \sum\limits_i e^{\jmath \beta_i x} \frac{-1}{\mu_0\omega} \sum\limits_m  \gamma_m\left( g_{i,m}^{+} - g_{i,m}^{-} e^{\jmath \gamma_m d} \right)
\end{array}	
\end{equation*}

Separating diffracted orders (multiply by $e^{\jmath\beta_{i'} x}$ and integrate over $\int\limits_{-\infty}^{+\infty}dx$) gives

\begin{equation*}
\label{eq:rcwa_Maxwell_1} 
\begin{array}{l}
\delta_{i0}+ R_i = \sum\limits_m  \left( g_{i,m}^{+} + g_{i,m}^{-} e^{\jmath \gamma_m d} \right)  \\\\
Y_{I,0}\delta_{i0} - R_i  Y_{I,i} = \frac{-1}{\mu_0\omega}\sum\limits_m   \gamma_m\left( g_{i,m}^{+} - g_{i,m}^{-} e^{\jmath \gamma_m d} \right)
\end{array}	
\end{equation*}

This equation can be written in matrix form by further assuming $g_{i,m}^{\pm} = v_{i,m} c_m^{\pm} $,  and $w_{i,m}^{\pm} = -1/(\mu_0\omega)\gamma_m v_{i,m}^{\pm}$

\begin{empheq}[box={\mymath[colback=white!30,drop lifted shadow, sharp corners]}]{equation}
\label{eq:boundaryR} 
\begin{array}{l}
\delta_{i0}+ R_i = \sum\limits_m  v_{i,m}\left( c_{m}^{+} + c_{m}^{-} e^{\jmath \gamma_m d} \right)  \\\\
Y_{I,0}\delta_{i0} - Y_{I,i}R_i  = \sum\limits_m  w_{i,m}\left( c_{m}^{+} - c_{m}^{-} e^{\jmath \gamma_m d} \right)
\end{array}	
\end{empheq}

%==========================================
%==========================================
\subsection{Boundary condition at z=d}

\begin{equation*} 
\label{eq:rcwa_Maxwell_1} 
\begin{array}{l}
\sum\limits_i e^{\jmath \beta_i x} \sum\limits_m v_{i,m} \left[ c_{m}^{+} e^{\jmath \gamma_m d} + c_{m}^{-}  \right] = \sum\limits_i T_i e^{\jmath\beta_i x}  \\\\
\sum\limits_i e^{\jmath \beta_i x}  \sum\limits_m w_{i,m} \left[ c_{m}^{+}  e^{\jmath \gamma_m d} - c_{m}^{-}   \right] = \sum\limits_i T_i  Y_{II,i} e^{\jmath\beta_i x}
\end{array}	
\end{equation*}

Separating diffracted orders (multiply by $e^{\jmath\beta_{i'} x}$ and integrate over $\int\limits_{-\infty}^{+\infty}dx$) gives

\begin{empheq}[box={\mymath[colback=white!30,drop lifted shadow, sharp corners]}]{equation}
\label{eq:boundaryT} 
\begin{array}{l}
\sum\limits_m v_{i,m} \left( c_{m}^{+} e^{\jmath \gamma_m d} + c_{m}^{-}  \right) = T_i   \\\\
\sum\limits_m w_{i,m} \left( c_{m}^{+} e^{\jmath \gamma_m d} - c_{m}^{-}  \right) = T_i  Y_{II,i} 
\end{array}	
\end{empheq}


%==========================================
%==========================================
\subsection{Eigen values}  

eliminating $R_i$ from \ref{eq:boundaryR} and $T_i$ from \ref{eq:boundaryT} gives 

\begin{empheq}[box={\mymath[colback=white!30,drop lifted shadow, sharp corners]}]{equation}
\begin{array}{l}
\left( Y_{I,0} +  Y_{I,i}\right) \delta_{i0}  - Y_{I,i} \sum\limits_m v_{i,m} \left( c_{m}^{+} + c_{m}^{-} e^{\jmath \gamma_m d} \right) = \sum\limits_m w_{i,m} \left( c_{m}^{+} - c_{m}^{-} e^{\jmath \gamma_m d} \right)\\\\
Y_{II,i} \sum\limits_m v_{i,m} \left[ c_{m}^{+} e^{\jmath \gamma_m d} + c_{m}^{-}  \right]  = \sum\limits_m w_{i,m} \left[ c_{m}^{+} e^{\jmath \gamma_m d} - c_{m}^{-}  \right] 
\end{array}
\label{eq:noR}
\end{empheq}


  
Which can be written in a matrix form assuming $V$, $W$, $X$, $Y_I$, and $Y_{II}$, are diagonal matrices with elements $v_{i,m}$, $w_{i,m}$, $\exp(\jmath\gamma_m d)$, $Y_{I,i}$, $Y_{II,i}$:  

 
\begin{empheq}[box={\mymath[colback=white!30,drop lifted shadow, sharp corners]}]{equation}
\begin{array}{l}
\begin{bmatrix}
Y_{I}V & Y_{I}VX \\
Y_{II}VX   & Y_{II}V
\end{bmatrix}
\begin{bmatrix}
c^+ \\
c^-
\end{bmatrix}
=
\begin{bmatrix}
-W & W X \\
W X  & -W
\end{bmatrix}
\begin{bmatrix}
c^+ \\
c^-
\end{bmatrix}
+
\begin{bmatrix}
\left[\left( Y_{I,0} +  Y_{I,i}\right) \delta_{i0}\right]_{n\times1} \\
\left[0\right]_{n\times1}
\end{bmatrix}	
\end{array}	
\label{eq:eig}
\end{empheq}
 
 or
 
 \begin{empheq}[box={\mymath[colback=white!30,drop lifted shadow, sharp corners]}]{equation}
\begin{array}{l}
\begin{bmatrix}
c^+ \\
c^-
\end{bmatrix}
=
\begin{bmatrix}
Y_{I}V+W & (Y_{I}V-W)X \\
(Y_{II}V-W)X   & Y_{II}V+W
\end{bmatrix}^{-1}
\begin{bmatrix}
\left[\left( Y_{I,0} +  Y_{I,i}\right) \delta_{i0}\right]_{n\times1} \\
\left[0\right]_{n\times1}
\end{bmatrix}	
\end{array}	
\label{eq:eig}
\end{empheq}
 


\nocite{Smith:2012qr}
%\nocite{*}



%----------------------------------------------------------------------------------------
%	BIBLIOGRAPHY
%----------------------------------------------------------------------------------------
\bibliographystyle{apalike}

\nocite{*}
\bibliography{sample}

%----------------------------------------------------------------------------------------


\end{document}
